\documentclass{rpt}

\title{Weekly Report}
\author{}
\date{\today}

\begin{document}

\maketitle

\section{Targets}

\subsection{Urgent}
\begin{itemize}
    \item
        \begin{color}{red}
            \sout{Change PI don't care set: simplify CNF by logic synthesis or special input pattern.}
        \end{color}
    \item
        \begin{color}{red}
            \sout{Change PI care set: use different random seed.}
        \end{color}
    \item
        \begin{color}{red}
            \sout{Change window care set.}
        \end{color}
    \item
        Change window don't care set.
\end{itemize}

\subsection{Important}
\begin{itemize}
    \item
        Randomly selecting a node,
        compute its care set,
        change its care set,
        synthesize the local circuit,
        evaluate its area and error rate,
        accept it with a certain probability.
    \item How DC affects BDD\@.
        Change simplification method (mfs$\rightarrow$bdd).
    \item Introduce DC with PLA files
    \item Use model counting to compute error rate, i.e., how many assignments satisfying a SAT problem.
    \item Use approximate confidence interval / hypothesis testing of Bernoulli experiments to evaluate the accuracy of error rate.
    \item Trade off the accuracy of batch error estimation for speed
        (even directly use Su's equation to update Boolean difference),
        perhaps use hypothesis testing to evaluate the accuracy.
    \item Combine the simulation of circuits with the simulation of Monte Carlo Tree Search.
        In other words,
        in one loop of Monte Carlo Tree Search,
        merge logic simulation and playout (only simulate circuit once and playout once).
    \item Represent circuit with AIG because of more potential LAC candidates.
        For each round, select one or more input wires and replace them with constant 0 or 1.
        Consider how to combine Wu's method (choose a subset of input wires and substitute).
    \item Accelerate Approximate Logic Synthesis Ordered by Monte Carlo Tree Search:
        reuse the result of batch error estimation in playout.
    \item
        Use UCB1's bound to guide the simulation time.
        Find relationship of different bounds.
\end{itemize}

\subsection{Worth Trying}
\begin{itemize}
    \item Enhance default policy with greedy approach or field domain knowledge.
    \item In expansion process of MCTS, expand more than one layers.
    \item Tune parameters in MCTS\@.
    \item Perform greedy flow on leaves of the final Monte Carlo Search Tree.
    \item Combine beam search and MCTS\@.
    \item
        Influence of network representation on synthesis.
        Why does mfs use local AIG function to represent the circuit,
        is it more fittable to LUT mapping?
\end{itemize}

\subsection{Potential Topics}
\begin{itemize}
    \item Relationship between power simulation and logic simulation.
    \item Combine Binarized Neural Network with approximate computing.
    \item Relationship between Boolean network and Bayesian Network.
    \item Approximate TMR\@.
\end{itemize}

\section{Progress}

\subsection{Issues of Last Meeting}

\subsubsection*{Why area does not change with $M$ for a fixed $l$}
Recall the definition of $k$ and $l$:
$l$ is a relevant level.
For a node $V$,
we extract its local circuit $\mathbb{W}$ by limiting the level of transitive fanins/fanouts as $k$ and $l$.
% The level of each local input in $\mathbb{W}$ is at least $V.level - l$.

The issue is:
For a fixed $l$,
area of the generated approximate circuit did not change with the simulation number $M$.

I made a stupid mistake.
In my wrong code,
$M$ is set as a constant value 64.

\subsubsection*{The relationship between DCs on local inputs and primary inputs}
Need to be discussed.

\subsection{Issues of This Week}
\subsubsection*{Result of changing window care set}
Let $k=2$, $l=1$.
Set approximate care set on local inputs according to simulation values.
The result is in Table~\ref{tab:res}.
\begin{table}[!h]
\centering
\caption{Area and error versus simulation number}
\begin{tabular}{cccccc}
\toprule
& \multicolumn{3}{c}{Approximate windows care set}&\multicolumn{2}{c}{reference (Su TCAD)}\\
circuit&\#simulation&area&error&area&error upper bound\\
\midrule
c880&64&477&0.743945000&504&0.05\\
&128&404&0.839746000&&\\
&192&537&0.480078000&&\\
&256&519&0.470801000&&\\
&257&545&0.002343750&554&0.003\\
&320&545&0.002343750&&\\
&384&550&0.002148440&&\\
&448&559&0.000292969&569&0.001\\
&512&546&0.003515620&&\\
&576&559&0.000292969&&\\
&640&559&0.000292969&&\\
&704&573&0.002148440&&\\
&768&577&0.000292969&&\\
&832&581&0.000195313&&\\
&896&576&0.000097656&&\\
&960&559&0.000292969&&\\
&1024&580&0.00195312&&\\
&10240&591&0&599.00&0\\
c1908&1&717&0&284&0.05\\
&16&624&0.702344&&\\
&31&697&0.0775391&&\\
&32&706&0.00478516&619&0.005\\
&48&706&0.00478516&&\\
&64&710&0&&\\
\bottomrule
\end{tabular}\label{tab:res}
\end{table}

\subsubsection*{Problems in Table~\ref{tab:res}}
\begin{itemize}
\item Error rate changes dramatically near some values of simulation number,
such as 256$\rightarrow$ 257 for c880, 31$\rightarrow$32 for c1908.
\item Error rate dose not monotonous to simulation number.
\end{itemize}

The drawback of mfs method (only find one feasible re-substitution function) might be responsible for those.

\subsubsection*{Other problems}
The number of local circuit inputs is not balanced,
and it may result in large error rate for those local circuit with many inputs.

\end{document}
